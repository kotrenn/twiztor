\documentclass{report}

\usepackage{fullpage}

\begin{document}

\chapter{Background}

\section{Graph Theory}

An {\bf undirected graph} $G = (V, E)$ over a set of vertices $V$ and edges $E\subseteq V\times V$.
Note that $E$ should be a multisubset of $V\times V$.

\section{Group Theory}

\subsection{Permutation Groups}

Given a set $X$ of points, the {\bf symmetric group} on $X$, denoted $S_X$, is the group of all permutations on $X$.

\subsection{Coxeter Groups}

A group is a {\bf Coxeter group} if it is generated by a set of involutions.

Some examples of Coxeter twiztors include AAA, AFD, and AFG.

\subsection{Cayley Graphs}

Given a group $G$ and a set of generators $S$ for G, the {\bf Cayley graph} $C(G, S)$ is the directed graph with vertices the elements of $G$ and an edge from $g$ to $h$ if and only if $h = gs$ for some $s\in S$.
The {\bf expanded Cayley graph} is the undirected graph with edge $gh$ if and only if $g = sh$ or $h = sg$ for some generator $s\in S$.

For twiztors, the Cayley graph is not necessarily the same as the twiztor graph.

The puzzle ACG is based on the Cayley graph of $S_4$ with generators $S = \{(1 2 3), (1 4)\}$.

\chapter{Twiztor Properties}

Define a {\bf twiztor} $T = (X, C, P)$ to be a pair of a set $X$ of {\bf points}, a partition $C$ of $X$, and a set $P\subseteq S_X$ of permutation generators on $X$.

In general, we will only look at twiztors for which $X$ is finite.

\chapter{Puzzle Families}

Some subsets of twiztors in the database form interesting families.

\section{Graphs}

Every undirected graph can be converted to a Coxeter twiztor by way of an edge coloring of the graph.

If unique colors are used for every edge, then isomorphic to a sub-generator of $S_n$ with all possible inversions.

\subsection{Trees}

A ton of puzzles.
Number of non-isomorphic tree twiztors?

\subsubsection{Binary Trees}

Puzzles ADT, AFI, ACS.
Assumes edges map to distinct left/right partitions.

\section{Isometric Graphs}
Coxeter twiztors.

\subsubsection{Hex Graphs}
Bunch of them.

\section{Cycle Gluing}
Twiztors $C_m\cdot C_n$ and $C_m\circ C_n$.  Can be joined as a graph.

\section{Two Generators}

Arbitrary complexity of two generators.

Twiztors with generators which are disjoint cycles of length $3$ $\Rightarrow $ even permutations.
Cayley graph is isomorphic to line graph of a cubic graph.

\end{document}
